\section*{Bài 1.6}
\addcontentsline{toc}{section}{\protect\numberline{}Bài 1.6}%

\begin{problem}{1.6}
% Type problem here!!!
$\overline{X}_1$ và $S^2_1$ là trung bình mẫu và phương sai mẫu của một cỡ mẫu $n_1$ được chọn từ một tổng thể kỳ vọng $\mu_1$ và phương sai $\sigma_1^2$. Tương tự $\overline{X}_2$ và $S^2_2$ là trung bình mẫu và phương sai mẫu của một cỡ mẫu $n_2$ được chọn từ một tổng thể thứ 2 độc lập với tổng thể thứ nhất có kỳ vọng $\mu_2$ và phương sai $\sigma_2^2$
\begin{enumerate}[label=(\alph*)]
\item Chứng tỏ rằng $\overline{X}_1$-$\overline{X}_2$ là một ước lượng không chệch cho $\mu_1$-$\mu_2$.
\item Tìm sai số chuẩn (Standard Error, tức là đọ lệch chuẩn của ước lượng) của $\overline{X}_1$-$\overline{X}_2$. Làm thế nào để ước lượng sai số chuẩn này
\item Giả sử phương sai hai tổng thể bằng nhau, tức là $\sigma_1^2$= $\sigma_2^2$=$\sigma^2$. Chứng tỏ rằng:
\[ S^2_p = \dfrac{(n_1-1)S^2_1+(n_2-1)S^2_2}{n_1+n_2-2}\]
là một ước lượng không chệch của  $\sigma^2$.
\end{enumerate}
\end{problem}

\begin{proof}
(a)\\
Ta biết $\overline{X}_1$, $\overline{X}_2$ lần lượt là ước lượng không chệch của $\mu_1$ và $\mu_2$, nên:
\begin{equation*}
{E(\overline{X}_1-\overline{X}_2) = E(\overline{X}_1)-E(\overline{X}_2) = \mu_1-\mu_2}
\end{equation*}
Vậy $\overline{X}_1$-$\overline{X}_2$ là một ước lượng không chệch cho $\mu_1$-$\mu_2$.\\
(b)\\
Ta có:
\begin{equation*}
Var(\overline{X_1}-\overline{X_2}) = Var(\overline{X_1})-Var(\overline{X_2})
\end{equation*}
\begin{equation*}
\Rightarrow Var(\overline{X}_1-\overline{X}_2) = Var(\overline{X}_1)-Var(\overline{X}_2) =  Var(\frac1n_1\sum_{i=0}^{n_1}X_i)-Var(\frac1n_2\sum_{j=0}^{n_2}X_j)
\end{equation*}
\begin{equation*}
\Rightarrow Var(\overline{X}_1-\overline{X}_2) = \frac{1}{n^2_1}Var(x_1X_i)-\frac{1}{n^2_2}Var(n_2X_j) = \frac{1}{n_1}S^2_1-\frac{1}{n_2}S^2_2 
\end{equation*}
Vậy độ lệch chuẩn của ước lượng $\overline{X}_1$-$\overline{X}_2$ là $\sqrt{\frac{1}{n_1}S^2_1-\frac{1}{n_2}S^2_2}$\\
(c)\\
Ta có:
\begin{equation*}
S^2_p = \dfrac{(n_1-1)S^2_1+(n_2-1)S^2_2}{n_1+n_2-2}
\end{equation*}
\begin{equation*}
\Rightarrow E(S^2_p) = E(\dfrac{(n_1-1)S^2_1+(n_2-1)S^2_2}{n_1+n_2-2})
\end{equation*}
\begin{equation*}
\Rightarrow E(S^2_p) = \dfrac{(n_1-1)E(S^2_1)+(n_2-1)E(S^2_2)}{n_1+n_2-2}
\end{equation*}
\begin{equation*}
\Rightarrow E(S^2_p) = \dfrac{(n_1-1)\sigma_1^2+(n_2-1)\sigma_2^2}{n_1+n_2-2}
\end{equation*}
\begin{equation*}
\Rightarrow E(S^2_p) = \dfrac{(n_1-1)\sigma^2+(n_2-1)\sigma^2}{n_1+n_2-2} 
\end{equation*}
\begin{equation*}
\Rightarrow E(S^2_p) = \sigma^2
\end{equation*}
Vậy $S^2_p$ là một ước lượng không chệch của  $\sigma^2$.
\end{proof}
