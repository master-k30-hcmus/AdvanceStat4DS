\subsection{Bài 1.10}
%\addcontentsline{toc}{section}{\protect\numberline{}Bài 1.10}%

\begin{problem}{1.10}
% Type problem here!!!
Xét $X$ là một biến ngẫu nhiên tuân theo phân phối mũ dịch cuyển có hàm mật độ xác suất cho bởi

\begin{equation} 
f\left(\left.x\right|\lambda,\theta\right) = \begin{cases} \lambda e^{-\lambda\left(x-\theta\right)} & , x\geq\theta\\0 &, \textrm{nơi khác} \end{cases} 
\end{equation} 
Hãy tìm một ước lượng hợp lý cực đại cho tham số $\lambda$ và $\gamma$ dựa trên mẫu ngẫu nhiên $\left(X_1,X_2,..,X_n\right)$
\end{problem}

\begin{proof}
% Type solution here!!!
Theo định nghĩa, ta xác định được hàm hợp lý $L\left(\gamma, \theta\right)$ với mẫu ngẫu nhiên $x = \left(x_1, x_2,...,x_n\right)$ là:
\[L\left(\left.\lambda,\theta\right|x\right)=\prod_{i=1}^nf\left(\left.x_i\right|\lambda,\theta\right)=\prod_{i=1}^n\lambda^ne^{-\lambda\left(x_i-\theta\right)}=\lambda^ne^{-\lambda\left(\sum_{i=0}^nx_i-n\theta\right)}\\L(\lambda,\theta)\]
Lấy log hai vế, ta được:
\begin{equation*}
\mathbb{L}(\lambda,\theta)=\ln\left(L\left(\left.\lambda,\theta\right|x\right)\right)=n\ln(\lambda)-\lambda\left(\sum_{i=0}^nx_i-n\theta\right)=n\ln\lambda-\lambda\sum_{i=0}^nx_i+\lambda n\theta
\end{equation*}
Ta đi tìm cực đại của hàm số $L(\lambda,\theta)$ theo hai biến $\lambda$ và $\theta$ bằng cách tính đạo hàm từng phần
\begin{equation*}
\nabla L=\begin{bmatrix}\frac{\partial L}{\partial\lambda}\\\frac{\partial L}{\partial\theta}\end{bmatrix}=\begin{bmatrix}\frac n\lambda-\sum_{i=0}^nx_i+n\theta\\\lambda n\end{bmatrix}
\end{equation*}
Ta đi tìm cực đại đối với biến $\lambda$:
\begin{equation*}
\frac n\lambda-n\sum_{i=0}^n\left(x_i-\theta\right)=0\
\end{equation*}
\begin{equation*}
\Rightarrow\widehat\lambda=\frac1n\sum_{i=0}^n\left(x_i-\theta\right)
\end{equation*}
Đối với biến $\theta$:
\begin{equation*}
\frac{\partial L}{\partial\theta}=\lambda n\geqslant0\\
\end{equation*}
Ta có thể được $\lambda n$ luôn dương, đồng nghĩa với việc khi tìm giá trị cực đại của biến $\theta$, ta sẽ chọn $\theta$ càng lớn càng tốt $\left( \theta \rightarrow \inf \right)$. Tuy nhiên, ta có điều kiện chặn trên $\theta \leqslant x $, suy ra được giá trị $\theta$ ta có thể là giá trị nhỏ nhất của biến ngẫu nhiên $X$, tức là:
\begin{equation*}
\hat \theta = min(X)
\end{equation*}
Vậy ta đã tìm được 2 ước lượng hợp lý cực đại của 2 tham số $\lambda, \theta$
\end{proof}